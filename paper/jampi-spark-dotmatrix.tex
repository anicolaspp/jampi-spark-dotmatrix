
%----------------------------------------------------------------------------------------
%	PACKAGES AND OTHER DOCUMENT CONFIGURATIONS
%----------------------------------------------------------------------------------------
\pdfoutput=1
\documentclass[fleqn,10pt]{SelfArx} % Document font size and equations flushed left

\usepackage[english]{babel} % Specify a different language here - english by default

\usepackage[super,biblabel]{cite} % Superscript citations

\usepackage{setspace} % for block quoting

\usepackage{etoolbox}

\usepackage{graphicx}

\usepackage{algpseudocode}

\AtBeginEnvironment{quote}{\singlespace\vspace{-\topsep}\small}
\AtEndEnvironment{quote}{\vspace{-\topsep}\endsinglespace}

%----------------------------------------------------------------------------------------
%	COLUMNS
%----------------------------------------------------------------------------------------

\setlength{\columnsep}{0.55cm} % Distance between the two columns of text
\setlength{\fboxrule}{0.75pt} % Width of the border around the abstract

%----------------------------------------------------------------------------------------
%	COLORS
%----------------------------------------------------------------------------------------

\definecolor{color1}{RGB}{0,0,90} % Color of the article title and sections
\definecolor{color2}{RGB}{0,20,20} % Color of the boxes behind the abstract and headings

%----------------------------------------------------------------------------------------
%	HYPERLINKS
%----------------------------------------------------------------------------------------

\usepackage{hyperref} % Required for hyperlinks
\hypersetup{hidelinks,colorlinks,breaklinks=true,urlcolor=color2,citecolor=color1,linkcolor=color1,bookmarksopen=false,pdftitle={Title},pdfauthor={Author}}

%----------------------------------------------------------------------------------------
%	ARTICLE INFORMATION
%----------------------------------------------------------------------------------------

% \JournalInfo{..., 2020} % Journal information
% \Archive{arXiv pre-print} % Additional notes (e.g. copyright, DOI, review/research article)

\PaperTitle{JAMPI: efficient matrix multiplication in Spark using Barrier Mode} % Article title

\Authors{Tamas Foldi\textsuperscript{1}*, Chris von Csefalvay\textsuperscript{1}} % Authors
\affiliation{\textsuperscript{1}\textit{Starschema Inc., Arlington, VA.}} % Author affiliation
\affiliation{*\textbf{Corresponding author}: tfoldi@starschema.net} % Corresponding author

% \Keywords{Spark --- Matrix multiplication --- Algorithmic methods} % Keywords - if you don't want any simply remove all the text between the curly brackets
% \newcommand{\keywordname}{Keywords} % Defines the keywords heading name

%----------------------------------------------------------------------------------------
%	ABSTRACT
%----------------------------------------------------------------------------------------

\Abstract{The new barrier mode in Apache Spark allows embedding distributed deep learning training as a Spark stage to simplify the distributed training workflow. In Spark, a task in a stage doesn’t depend on any other tasks in the same stage, and hence it can be scheduled independently. However, several algorithms require more sophisticated inter-task communications, similar to the MPI paradigm. By combining distributed message passing  (using nio/netty), JVM's new \texttt{Vector<>} API and Spark's barrier mode, we can add non-map/reduce based algorithms, such as Cannon's distributed matrix multiplication, improving the performance of the existing MLlib implementation. This paper discloses an efficient distributed matrix multiplication algorithm within a barrier task, which results in an XXX\% performance increase on a 10,000x10,000 square matrix. Applications of efficient matrix multiplication include significantly accelerating the training and implementation of deep convolutional neural network based workloads.}

%----------------------------------------------------------------------------------------

\begin{document}
%\flushbottom % Makes all text pages the same height
\maketitle % Print the title and abstract box
%\tableofcontents % Print the contents section
% \thispagestyle{empty} % Removes page numbering from the first page

%----------------------------------------------------------------------------------------
%	ARTICLE CONTENTS
%----------------------------------------------------------------------------------------

\section{Introduction} % (fold)
\label{sec:introduction}

The matrix multiplication operation $\star$ for an $p \times q$ matrix $\mathbf{A}$ and an $q \times r$ matrix $B$ is defined so that for the resultant matrix $\mathbf{C} = \mathbf{A} \star \mathbf{B}$, each element $c_{i, j}$ is the dot product of the $i$-th row of $\mathbf{A}$ and the $j$-th row of $\mathbf{B}$, i.e.

$$ c_{i, j} = \sum_{k = 1}^n a_{i, k} b_{k, j} $$

Matrix multiplication plays a significant role in a range of practical applications, including (but not limited to) scientific computing, non-linear modelling, agent-based models and the training of deep convolutional neural networks (deep learning). Deep learning has become a crucial tool for both research and commercial applications, including in computer vision,\cite{guo2016deep,voulodimos2018deep}  bioinformatics,\cite{spencer2014deep,alipanahi2015predicting,zhang2016deep,wei2018prediction} natural language processing (NLP),\cite{deselaers2009deep,socher2012deep,young2018recent,otter2020survey} clinical medicine,\cite{bar2015chest,havaei2016deep,liu2017detecting,stead2018clinical,campanella2019clinical,lehman2019mammographic} anomaly detection in cybersecurity and fraud detection,\cite{du2017deeplog,shone2018deep,chalapathy2019deep} and collaborative intelligence/recommender systems.\cite{wang2015collaborative,deng2016deep,karatzoglou2017deep,batmaz2019review} The proliferation of deep learning as the cognitive technology of choice for problems with large source data sets and high-dimensional or high-order multivariate data means that efficiency gains in the underlying linear algebra primitives has the potential to enable significant performance benefits in a wide range of use cases. In particular, constructing primitives that leverage computational capacity through rapid parallel computation and efficient interchange lends itself as an avenue towards these performance gains. While packages comprising efficient matrix primitives already exist,\cite{chetlur2014cudnn} these often operate at a low level and do not integrate well with existing and proven solutions to manage large computational loads. In particular, few integrate with Apache Spark,\cite{zaharia2016apache} by far one of the most widely used large distributed cluster computing frameworks at the time of writing, and those that do are relatively less efficient.

JAMPI (Java Assisted Matrix Product with Inter-task communication), the framework described in this paper, is an efficient and rapid solution to an aspect of efficienty matrix primitives, namely matrix multiplication. By integrating JVM's new \texttt{Vector<>} API, \texttt{nio/netty} for distributed message passing and Spark's barrier mode, a pure Scala implementation of Cannon's 2.5D matrix multiplication algorithm can be devised that is significantly more efficient than \texttt{MLlib}'s \texttt{BlockMatrix.multiply} function. JAMPI thus avoids reliance on low level code on one hand, being a pure Scala implementation. On the other hand, it provides a pre-written framework that integrates with Spark as a native task rather than an external MPI procedure call, and handles inter-task communication directly, yielding performance benefits that would otherwise be associated with a low-level MPI implemented resource negotiation framework.

\subsection{Cannon's algorithm} % (fold)
\label{sub:cannon_s_algorithm}

The multiplication of square matrices constituites a special case. For a square matrix of order $n$, i.e. an $n \times n$ matrix, a special case obtains, which can be resolved efficiently using Cannon's algorithm.\cite{cannon1969cellular} 

For a square matrix of order $n$, i.e. $n \times n$, Cannon's algoritm uses a toroidally connected mesh $\mathjbf{P}^{n \times n}$ of $n^2$ processes. Rendered in pseudocode, the algorithm can be expressed as follows for $p$ processors:

\begin{algorithmic}
	\ForAll{i = 0 : \sqrt{p} - 1}
		\State CShift left A[i; :] by i 
	\EndFor
	
	\ForAll{j = 0 : \sqrt{p} - 1}
		\State CShift up B[:; j] by j 
	\EndFor
	
	\For{k = 0 : \sqrt{p} - 1}
		\For{i = 0 : \sqrt{p} - 1, j = 0 : \sqrt{p} - 1}
			\State C[i, j] += A[i, j] * B[i, j]
			\State CShift left A[i; :] by 1
			\State CShift up B[:; j] by 1
		\EndFor
	\EndFor
\end{algorithmic}

Cannon's algorithm is designed to be performed on a virtual square grid $\mathbf{P}$ of $p$ processors (i.e. a $\sqrt{p} \times \sqrt{p}$ matrix). The multiplicand and multiplier matrices $\mathbf{A}$ and $\mathbf{B}$ are laid out on $\mathbf{P}$, after which the $i$-th row of $\mathbf{A}$ is circularly shifted by $i$ to the left and the $j$-th column of $\mathbf{B}$ circularly shifted by $j$ elements up. Then, $n$ times, the two entries mapped onto $p_{i, j}$ are multiplied and added onto the running value of $p_{i, j}$, after which each row of $\mathbf{A}$ is shifted left by one element and each column of $\mathbf{B}$ is shifted up by one element.

Standard methods of multiplying dense matrices require $O(n^3)$ floating operations for an $n \times n$ matrix. Cannon's algorithm improves on this by reducing it to $O(\frac{n^3}{p})$. In particular, because of the fact that memory is not dependent on the number of processors, it scales dynamically with the number of processors. This makes it an attractive candidate for implementation as a high-performance distributed matrix multiplication primitive.

% subsection cannon_s_algorithm (end)

\subsection{Spark's barrier mode} % (fold)
\label{sub:spark_s_barrier_mode}

% subsection spark_s_barrier_mode (end)

% section introduction (end)

\section{Methods} % (fold)
\label{sec:methods}

\subsection{Cannon's algorithm on MPI} % (fold)
\label{sub:cannon_s_algorithm_on_mpi}

% subsection cannon_s_algorithm_on_mpi (end)

\subsection{Matrix multiplication as a barrier task} % (fold)
\label{sub:matrix_multiplication_as_a_barrier_task}

% subsection matrix_multiplication_as_a_barrier_task (end)

\subsection{Vector unrolling} % (fold)
\label{sub:vector_unrolling}

% subsection vector_unrolling (end)

\subsection{Inter-node communication} % (fold)
\label{sub:inter_node_communication}

% subsection inter_node_communication (end)

\subsection{Test protocols} % (fold)
\label{sub:test_protocols}

% subsection test_protocols (end)

% section methods (end)

\section{Results} % (fold)
\label{sec:results}

% section results (end)

\section{Conclusion} % (fold)
\label{sec:conclusion}

% section conclusion (end)

%------------------------------------------------
\phantomsection
\section*{Acknowledgments} % The \section*{} command stops section numbering

% \addcontentsline{toc}{section}{Acknowledgments} % Adds this section to the table of contents

The authors wish to thank Anjan Banerjee for the discussions that inspired this paper. All errors and ommissions are the authors' own.

\phantomsection
\section*{Competing interests} % The \section*{} command stops section numbering

% \addcontentsline{toc}{section}{Acknowledgments} % Adds this section to the table of contents

The authors have declared no competing interest.

\phantomsection
\section*{Funding statement} % The \section*{} command stops section numbering

% \addcontentsline{toc}{section}{Acknowledgments} % Adds this section to the table of contents

The research summarised in this paper was funded by Starschema Inc.


%----------------------------------------------------------------------------------------
%	REFERENCE LIST
%----------------------------------------------------------------------------------------
\phantomsection

\bibliographystyle{unsrt}
\bibliography{bibliography}

%----------------------------------------------------------------------------------------

\end{document}